
%% * Define the approach we will follow
%% * Recap v, q, pi
%% * Evaluating a policy: v^pi, q^pi
%%   * Define bellman expectation equations
%%   * Backup diagrams for Bellman Expectation Equations
%%   * Policy evaluation
%%   * Gridworld example
%% * Optimality: v*, q*
%%   * Define bellman optimality equations
%%   * Backup diagrams for Bellman Optimality Equations
%%   * Policy iteration
%%   * Gridworld example
%%   * Value iteration
%%   * Gridworld example
%%   * Generalized Policy Iteration

\subsection{Dynamic Programming} \title{Dynamic Programming} \author{} \date{}
\begin{frame}[plain,c]
    \titlepage
\end{frame}

\begin{frame}
    \frametitle{You might want to check these resources}
    Below there are some lectures and resources that I think you should check
    in more depth after this tutorial. I'll try to give the required high-level
    overview such that you will feel comfortable with these awesome resources.
    \begin{itemize}
        \item \href{https://youtu.be/Nd1-UUMVfz4}{David Silver - lecture 3 on Dynamic Programming}
        \item \href{https://youtu.be/hMbxmRyDw5M}{Hado Van Hasselt - lecture 3 on Dynamic Programming}
        \item \href{http://incompleteideas.net/book/RLbook2018.pdf}{Sutton and Barto RL book - chapter 4}
    \end{itemize}
\end{frame}

\begin{frame}
    \frametitle{Dynamic Programming (DP)}
    \pause
    \begin{itemize}
        \item Fibonacci?, Floyd Warshall?, Top Coder?, ... and various other problems
              can be solved using this approach.

        \pause

        \item Recall that in those the key component was memoization and recursive
              formulation.

        \pause

        \item These two properties will appear in various problems, and can be defined
              as follows:
              \pause
              \begin{itemize}
                \item Principle of Optimality (\textbf{That recursive formulation}): If a
                      problem presents this property, then the optimal solution can
                      be decomposed solutions to subproblems in a recursive way.

                \pause
                
                \item Overlapping subproblems (\textbf{That memoization technique}): The 
                      solutions to the subproblems can be cached and reused.
              \end{itemize}

    \end{itemize}
\end{frame}